\documentclass[titlepage, a4paper]{article}
\input{../mall/layout.tex}	% Importera generella layout-strukturer

% Information nödvändig för generella layout-strukturer
\newcommand{\LIPSredaktor}{Johannes Klasson}
\newcommand{\LIPSversion}{P1B}
\newcommand{\LIPSdokument}{High-Level Design Report}
\newcommand{\LIPSdokumenttyp}{High-Level Design Report}
\newcommand{\LIPSgranskatdatum}{2016-02-15}
\newcommand{\LIPSgranskare}{Johannes Klasson}
\newcommand{\LIPSgodkannare}{Martin Nielsen-Lönn}
\newcommand{\LIPSgodkantdatum}{-}
\newcommand{\LIPSkursnamn}{TSEK06}
\newcommand{\LIPSprojektnamn}{16-bit Kogge-Stone Adder}
\newcommand{\LIPSprojektgrupp}{Group 5}
\newcommand{\LIPSartaltermin}{VT, 2016}
\newcommand{\LIPSkund}{ISY}
\newcommand{\LIPSkundkontakt}{Martin Nielsen-Lönn}
\newcommand{\LIPSkursansvarig}{Atila Alvandpour}
\newcommand{\LIPShandledare}{Martin Nielsen-Lönn}

% Dokument-specifika paket
\usepackage{tabularx}
\usepackage{pdfpages}
\usepackage{tikz}
\usepackage{float}
\usetikzlibrary{shapes, arrows}
\usepackage{booktabs} % Horizontal rules in tables
\usepackage[justification=centering]{caption}
\usepackage{adjustbox}
\pagenumbering{roman}


\DeclareGraphicsRule{.0.pdf}{pdf}{*}{}

\begin{document}

\LIPSTitelsida

\begin{LIPSprojektidentitet}
  \LIPSgruppmedlem{Johan Isaksson}{Project Leader}{070-2688785}{johis024@student.liu.se}
  \LIPSgruppmedlem{Johannes Klasson}{Document Manager}{073-8209003}{johkl226@student.liu.se}
  \LIPSgruppmedlem{Jonas Tarasso}{Designer}{070-5738583}{jonta760@student.liu.se}
  \LIPSgruppmedlem{Alexander Yngve}{Designer}{076-2749762}{aleyn573@student.liu.se}	
\end{LIPSprojektidentitet}

\newpage
\tableofcontents	%Innehållsförteckning
%\listoffigures
%\listoftables

\newpage

\begin{LIPSdokumenthistorik}
\LIPSversionsinfo{P1A}{2016-02-15}{First draft}{Johan Isaksson}
\end{LIPSdokumenthistorik}

\newpage
\pagenumbering{arabic}	%Påbörja sidnumrering
\section{Introduction}
\section{High level description}
\subsection{SPI/PSRBR}

\subsection{16-bit Kogge-Stone Adder}
\begin{table}[H]
\caption{Example table boolean expressions for combinatorial gates}
\centering
\begin{tabular}{cccc}
\toprule
Input & Function1 & Function2 & Output \\
\midrule
00 & 1 & 0& 1\\
01 & 0 & 0 & 1\\
10 & 0 & 0 & 1\\
11 & 1 & 0 & 1\\
\bottomrule
\label{tab:SPI}
\end{tabular}
\end{table}

\begin{figure}[H]
  \centering
  \captionsetup{justification=centering}
  \includegraphics[clip,width=1.0\textwidth]{../figures/red}
  \caption{Figure shows whats inside the red block in the adder} \label{fig:red}
\end{figure}
\subsection{Comparator}

\begin{figure}[H]
  \centering
  \captionsetup{justification=centering}
  \includegraphics[clip,width=1.0\textwidth]{../figures/yellow}
  \caption{Figure shows whats inside the yellow block in the adder} \label{fig:yellow}
\end{figure}

\begin{figure}[H]
  \centering
  \captionsetup{justification=centering}
  \adjustbox{trim={0\width} {0\height} {0\width} {.4\height},clip}
  {\includegraphics[width=1.0\textwidth]{../figures/yellow_carry}}
  \caption{Figure shows whats inside the yellow carry block in the adder} \label{fig:yellow_c}
\end{figure}

\begin{figure}[H]
  \centering
  \captionsetup{justification=centering}
  \includegraphics[clip,width=1.0\textwidth]{../figures/sum}
  \caption{Figure shows whats inside the sum block in the adder} \label{fig:sum}
\end{figure}


\end{document}
