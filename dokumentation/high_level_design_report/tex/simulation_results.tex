\section{Simulation Results} \label{sec:simulation_results}
This section describes the high level simulation results. All files referenced to in this section can be found in the attached zip-file.
\subsection{SPI In}
The first thing to test in the system is where it all begins, at the input. The basics of it can be seen in test\_spi\_receive. As can be seen, as soon as the SPI\_enable signal goes low and the SPI\_clk starts we start to receive one bit on every positive edge. The data is then shifted trough all of the 16 registers. As the 16th bit is shifted in, the first load signal is triggered. Every 16 bits after that another load signal is triggered, and this can be seen in test\_spi\_load. One can also see that once the last load signal is triggered, SPI\_enable goes high again.\\
The last thing in the SPI\_in module to test is how the data travels out of the PRBS registers. This can be seen in test\_spi\_prbs. One important thing to note is that as soon as SPI\_enable goes high, the registers are triggered on the system clock. As one can see in the figure, the first bit is ready for a long time, and as soon as the last bit is ready, we start to add at full speed. And after the four bits are done the system continues to add the pseudo random numbers. 

\subsection{Kogge-Stone Adder}
The simulation of the adder can be seen in test\_koggeadder and the input sequence is the same that were fed into the SPI. The most relevant part of the simulation is precisely after the topmost signal goes high. When this happens the data is already fed into the register in front of the adder and begins to shift into the adder on positive clock edge. The out, or to makes things more clear, the BISTout signal clearly shows that the two first addition yields the correct result but the thirds makes the same signal go low. This is a construction of the input from our side to test if the comparator can detect errors. After this the BISTout are low for a while before it goes high again which means that the fourth addition was successful.

\subsection{Comparator}
The simulation of the comparator is seen in test\_corr. The same reasoning as in the section above applies here. The simulation is quite striped down due to readability but one can clearly see that out, which is BISTout, is high if and only if the corr-signals and sum signals match.

\subsection{SPI Out}
The critical parts of this module are the events after a transition on the spi enable signal.\\


In the image sim\_spi\_en\_high, a simulation of the module when the spi enable signal goes high is shown. As can be seen, the control logic successfully creates the four enable pulses for the different sections. The four clock cycles long pulse prevents the occurrence of more than one pulse on any of the enable signals. \\



%\begin{figure}[H]
%\centering
%\captionsetup{justification=centering}
%\includegraphics[scale=0.2]{../figures/spi_en_high.png}
%\caption{Control logic for spi output, second half}
%\label{spi_en_h}
%\end{figure}


In the image sim\_spi\_en\_low a simulation of the module when spi enable goes low is shown. As can be seen, the four enable signals are the same as the spi clock.

%\begin{figure}[H]
%\centering
%\captionsetup{justification=centering}
%\includegraphics[scale=0.2]{../figures/spi_out_low.png}
%\caption{Control logic for spi output, second half}
%\label{spi_en_l}
%\end{figure}

\subsection{Top Level}
One can get a overall picture of the behaviour of the system by looking at at the simulations in the order spi\_receive, spi\_prbs, koggeadder, corr and last the spi\_out. This is possible because all this simulations are part of a bigger one.
