\section{Introduction}
This document describes the state of the 16-bit Kogge-Stone adder project in the course TSEK06 after finishing the high level design phase. The system itself is supposed to receive two numbers that should be added together. The result should then both sent out from the system and be compared with a checksum in a BIST (Built-In Self-Test). The meaning of high level is that every basic logic gate is implemented in Verilog-A. The main reason for doing this is to be able to simulate all logic to make sure that everything works as intended. Block level diagrams can be found in section \ref{sec:block_level}, simulation results in section \ref{sec:simulation_results} and a risk analysis in section \ref{sec:risks}. In appendix \ref{app:time_plan} and \ref{app:time_report} a time plan of the next phase and a time report of this phase can be found.
