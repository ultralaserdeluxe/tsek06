\section{Block Level Description} \label{sec:block_level}
Much of the block level descriptions can be seen in the high level report, but the transistor view of the leaf-cells will be described in this chapter. To find good sizes for our gates we used a very simple sizing strategy. Start small, and if the signal is to weak to drive the components, we just size it up and if necessary, make a buffer for it. The transistor schematic of the basic blocks like AND, OR, DFF etc. are simple enough that we will not include any description for them. In Fig. \ref{top} an updated block diagram of the complete system can be seen.

\begin{figure}[H]
\centering
\captionsetup{justification=centering}
\includegraphics[scale=0.175]{../figures/top_level.png}
\caption{Top level block diagram.}
\label{top}
\end{figure}

\noindent The updated diagram contains additional registers for synchronizing the signal before and after the comparator. This was done to provide more stable signals to the comparator and to have a more easily interpreted BISTout signal. The drawback is that the BISTout signal is available two clock cycles after the addition took place.   
