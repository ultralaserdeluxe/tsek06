\section{Chip evaluation}
This chapter will describe how to verify the functionality of the chip using a FPGA board and an oscilloscope.

\subsection{Pad list}
Tab. \ref{tab:pins} shows the pin assignments for the chip. The pad frame with corresponding names and pin type is shown in Fig. \ref{fig:padframe}. Fig. \ref{fig:setup} shows how to connect the chip to the FPGA board and oscilloscope.

\begin{table}[H]
  \caption{Pin assignments}
  \centering
  \begin{tabularx}{\linewidth}{|l|l|l|X|}
    \hline
    \textbf{Name} & \textbf{Direction} & \textbf{Type} & \textbf{Description} \\ \hline
    Vdd1 & INOUT & Analog & Will provide most of the system with power and will be a steady \SI{3.3}{\volt}. \\ \hline
    Vdd2 & INOUT & Analog & Will provide the adder with power and it might vary from \SI{3.3}{\volt} downto threshold-voltage. \\ \hline
    GND &  INOUT & Analog & Ground. \\ \hline
    Clk & IN & Analog & This is the clock for the adder, some registers and control logic. Should have a frequency of at least \SI{200}{\mega\hertz} at \SI{3.3}{\volt}. Will be lower as we decrease the voltage of Vdd2. \\ \hline
    SPI\_clk & IN & Digital & This clock is used by the input and output unit and should be at least four times slower than the system clock. Should also be low if SPI\_en is inactive. \\ \hline
    SPI\_en & IN & Digital & Active low. Should go high on the first negative flank of SPI\_clk after the last value is read. \\ \hline
    SPI\_in & IN & Digital & Updates it's value as soon as SPI\_en goes low, and should have it's value ready on the first positive flank of SPI\_clk, since this is when we read the value. The value of SPI\_in should then be updated on every negative flank of SPI\_clk. \\ \hline
    SPI\_out & OUT & Digital & The data is available for read on the first falling edge of SPI\_clk after SPI\_en has gone active. \\ \hline
    BISTout & OUT & Digital & If the IN-data is correct, BISTout should be constant high after the first addition is done until the the PRBS-bit is set.  \\ \hline
    Cin & IN & Digital & Used to measure propagation delay through the adder. \\ \hline
    Cout & OUT & Analog & Used to measure propagation delay through the adder. \\ \hline
    Sum15 & OUT & Analog & Used to measure propagation delay through the adder. \\ \hline
  \end{tabularx}
  \label{tab:pins}
\end{table}



 \newpage
\subsection{SPI Interface}
This section describes the protocol used to communicate with the chip.

\begin{figure}[H]
\centering
\captionsetup{justification=centering}
\includegraphics[scale=0.35]{../figures/spi_data.png}
\caption{SPI data diagram}
\label{data}
\end{figure}

As an attempt to save time on implementation, the group decided to deviate from the standard spi protocol and create an specialized protocol for the chip.\\
The total amount of data that should be sent, during each active period of SPI_enable, is 4 x 4 16 bit words to the chip and four 17 bit words from the chip. This is illustrated in Fig. \ref{spi_data}.
The chip receive and send data with the most significant bit first. The order of the bits is illustrated Fig. \ref{spi_int}. \\

\begin{figure}[H]
\centering
\captionsetup{justification=centering}
\includegraphics[scale=0.35]{../figures/spi_interface.png}
\caption{SPI timing diagram}
\label{spi_int}
\end{figure}

Data is both written and read by the chip on the rising edge of the the spi clock. This means that the circuitry communicating with the chip has to read and write on the falling edge of the clock. The first input data should be available on the first rising edge after SPI_en goes low, and the first bit of the output will be available on the first falling edge after spi enable goes low. This can also be seen in Fig. \ref{spi_int}

\begin{figure}[H]
\centering
\captionsetup{justification=centering}
\includegraphics[scale=0.35]{../figures/MUX_DFF.png}
\caption{Shift register cell}
\label{mux_dff}
\end{figure}
\newpage

Tab. \ref{tab:test_data} shows the data that should be fed into the chip to test it's functionality.

\begin{table}[H]
  \caption{Test data}
  \centering
  \begin{tabularx}{\linewidth}{|X|X|X|X|}
    \hline
    \textbf{A} & \textbf{B} & \textbf{Config bits} & \textbf{Check sum} \\ \hline
    1111111111111111 & 1111111111111111 & 0000000000000011 & 1111111111111111\\ \hline
    0100010000110001 & 0001000100100011 & 0000000000000000 & 0101010101010100\\ \hline
    1111111111111111 & 0000000000000000 & 0000000000000111 & 0000000000000000\\ \hline
    0000000000000000 & 0000000000000000 & 0000000000000100 & 0000000000000000\\ \hline
  \end{tabularx}
  \label{tab:test_data}
\end{table}   

\subsection{Evaluation}
To test the chip everything should be connected as shown in Fig. \ref{fig:setup}. The FPGA shown should be programmed to send out the data shown in Tab. \ref{tab:test_data} with the MSB first to be able to compare the results obtained during simulations before tape-out. The SPI clock should be at least four times slower than the system clock for the SPI-unit to function properly. While the data is being transferred into the chip clk\_enable should be held low, which is for 256 SPI clock pulses. After this clk\_enable must go high within 15 clock pulses to avoid overwriting previously transferred data.

\begin{figure}[H]
\centering
\captionsetup{justification=centering}
\includegraphics[scale=0.1]{../figures/padframe.png}
\caption{Overview of the pad frame.} \label{fig:padframe}
\end{figure}

The chip will be tested in two different modes namely, PRBS- and TEST-mode. The PRBS-mode is used to measure the power consumption of the adder while being fed with pseudo random data at full speed. TEST-mode on the other hand is used to measure the propagation delay through the adder. The propagation delay can be determined by measuring the time between transition on Cin to transition on Cout. For this to happen the input data must be chosen such that the adder overflows which causes Cout to go high. This happens for data number for in Tab. \ref{tab:test_data}. While in TEST-mode it's possible to test how fast the system can be clocked by looking at BISTout while ramping up the frequency. As long as BISTout is high the system functions as intended but as soon as BISTout goes low something failed to keep up with the clock.

\begin{figure}[H]
\centering
\captionsetup{justification=centering}
\includegraphics[scale=0.07]{../figures/evaluation_setup.png}
\caption{Pad frame showing how to connect things.} \label{fig:setup}
\end{figure} 