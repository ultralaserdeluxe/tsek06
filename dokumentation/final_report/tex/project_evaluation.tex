\section{Project Evaluation} \label{sec:evaluation}
During the course of the project we gained a lot of experience related to tools, design and also cooperation. 

\subsection{Cooperation} \label{sec:eval_coop}
The group has functioned very well during the whole project except for two weeks during the beginning of the layout phase. There were some miscommunication since half of the group were on holiday the first week while the other half were on holiday during the second week of this period. This later led to some integration issues.

\subsection{Tools}
In retrospect, less time should have been spent on VerilogA simulations and more time on schematic simulations. The VerilogA models were very inaccurate and we had no idea what values to choose for propagation delays and rise times et.c. The schematic simulations were much more accurate and at the same time quite fast. Since the simulations when doing layout were very slow, a lot of time was spent on waiting for results. Doing more design work at the schematic level would have led to more effective work and possibly a better overall design.

Another tool related issue encountered was the problem with configurations. Using configurations in Cadence seemed like a very useful feature which could help us keep a clean structure of the project by having several schematics and layouts of different versions of the same cell in one place, instead of creating several very similar cells. It turned out however that it was very hard to maintain and the configuration editor in Cadence randomly crashed all the time when creating somewhat advanced configurations. We should have stuck to having only one schematic and one layout view in each cell, even though this led to many cells.

Quite early in the project we researched the possibility of having automatic tests of all cells in the project using a tool called Ocean. Unfortunately we could not get it to work, but it would have been very useful and could have helped us pinpoint errors with ease.

\subsection{Design}
The design work went smoothly until the integration phase. We expected some hardships during integration, but maybe not to the extent we experienced them. Things seemed to stop working for no apparent reason. Cells that worked yesterday, didn't function today and so on. Integration is hard, this is a lesson all of us will remember. 

Looking back, there are several things we could have done much better. Clock distribution, interconnects and floor planning should have been considered much earlier in the project, during the same time as when layouting the basic gates. Floor planning and interconnects could have been done much better if we would have communicated better during the early layout phase, but due to the reasons mentioned in \ref{sec:eval_coop} this didn't happen. This led to a situation were individual parts functioned well but weren't optimally adapted to eachother. For example, some components of the chip had different widths and therefore the power rails and data interconnects couldn't be aligned easily. If the floor planning had been done earlier in the layout phase, we could also have designed the basic blocks better. When we designed the basic blocks we did it almost exclusivly using metal 1 and 2 layers, in order to leave room for power supply and clock signals on metal 3 and 4 layers. If we would have considered the power supply and clock signals at the same time as when we designed the basic blocks, we could probably have ended up with a tighter design.

Late in the project we discovered some timing issues probably related to the SPI control logic, but we didn't have time to make a new easier and faster control logic design since the layout was already done and the simulations took considerable time. If we had done more testing in the schematic stage, with more realistic loads and possible doing worst case corner testing in schematic simulations, this problem would have been discovered earlier and the control logic could have been redesigned. The problem was eventually solved without a new design but not in an optimal way.
