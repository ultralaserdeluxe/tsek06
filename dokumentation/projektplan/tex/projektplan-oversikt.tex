\section{Översiktlig beskrivning av projektet}


\subsection{Syfte och mål}
Syftet med projektet är att:
\begin{enumerate}
 \item Få djup insikt i fysik konstruktion av avancerade chip.
 \item Få kunskap och erfarenhet i användandet av professionella CAD verktyg för konstruktion, simulering, layout och verifiering av VLSI chip.
 \item Konstruera ett riktigt och fungerande chip från idé via beteendenivåmodellering till detaljerade kretskonstruktioner på transistornivå och slutligen layout och verifiering.
 \item Slutföra ett projektet på ett industriellt och professionellt sätt.
\end{enumerate}

\raggedright Målet med projektet är att konstruera en integrerad krets med hjälp av CMOS teknologi, i detta fall en 16-bitars Kogge-Stone adderare. 

\subsection{Leveranser}

%H fungerar inte utan float paketet
\begin{table}[H]
  \centering
    \begin{tabularx}{\textwidth}{| l | l | X |}
      \hline
      \textbf{Leverans} & \textbf{Ansvarig} & \textbf{Färdig} \\
      \hline

      {Högnivådesign klar och simulationsrapport inlämnad} & {Johan Isaksson} & {2016-02-19} \\
      \hline
      {Transistornivådesign klar och simulationsrapport inlämnad} & {Johan Isaksson} & {2016-03-18} \\
      \hline
      {Layout, LVS, DRC och parasitisk simulation ska vara färdig} & {Johan Isaksson} & {2016-05-18} \\
      \hline
      {Leverans av färdigt chip} & {Johan Isaksson} & {2016-05-23} \\
      \hline
      {Slutgiltlig rapport inlämnad och muntlig presentation} & {Johan Isaksson} & {2016-05-27} \\
      \hline

    \end{tabularx}
  \caption{Projektets leveranser.} \label{dokumentation:tabell}
\end{table}



\subsection{Begränsningar}
?