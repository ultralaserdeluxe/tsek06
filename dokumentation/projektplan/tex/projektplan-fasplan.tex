\section{Fasplan}
Projektet består av följande fem faser:
\begin{enumerate}
\item Förstudie
\item Högnivådesign
\item Transistornivådesign
\item Layout
\item Redovisning
\end{enumerate}

\subsection{Fas 1: Förstudie}
Under förstudien sker litteratursökning och fördjupning inom själv projektuppgiften, och en projektplan med tidplan skall levereras till handledare. 

\subsection{Fas 2: Högnivådesign}

Det första som skall göras under projektet är att utveckla en högnivådesign som matchar det beteende som kunden söker. Denna design kommer skrivas i ett HDL-språk, som i detta fallet troligtvis kommer vara Verilog, och designen ska simuleras för att sökerställa att den uppför sig enligt specifikation.

\subsection{Fas 3: Transistornivådesign}

Högnivådesignen från fas två är klar och börjar förfinas och mer detaljer läggs till. Allt eftersom mer detaljer läggs till så måste designen simuleras och verifieras igen. Detta är en iterativ process som går från blocknivå beskrivning genom en macromodelldesign ända ner till transistorimplementering av kretsen. Det huvudsakliga arbetet här kommer vara simuleringar. Om fel påträffas måste vi gå tillbaka till en högre nivå och ändra designen så att felen åtgärdas.

\subsection{Fas 4: Layout}

Här byggs allting från grunden. Små celler byggs med transistorer som sedan används i större block, och hela systemet byggs nedifrån upp. Efter varje steg på vägen upp så simuleras cellen för att säkerställa att den fungerar korrekt.

\subsection{Fas5: Redovisning}
Efter projektets slut skall en slutgiltlig rapport lämnas in, samt en presentation av projektet skall ske. 
