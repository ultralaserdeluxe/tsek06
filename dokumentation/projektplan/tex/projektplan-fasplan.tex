\section{Project phases}
The project consist of the following five phases:
\begin{enumerate}
\item Prestudy
\item High-level system description
\item Top-Down system description
\item Layout
\item Chip assembly
\end{enumerate}

\subsection{Phase 1: Prestudy}
During the prestudy phase there will be a lot of literature reading and getting a deeper understanding of the project task itself. We will also write the project plan including a time plan. 

\subsection{Phase 2: High-level system description}
The first step in the construction of the chip is to develop the high-level system description, which matches the behavior that the sponsor asked for. This will be done in a hardware descriptive language, which in this case will be Verilog-A, and the design will be simulated to verify that it behaves as intended. 

\subsection{Phase 3: Top-Down system description}
The high-level system description from phase 2 will now be refined to include further details. We will use a top-down methodology as we go from block-level description via gate-level down to a transistor-level implementation. This will be a iterative process as after each new detail we add, we need to simulate and verify the design again. This means that much of the work here will be simulations.

\subsection{Phase 4: Layout}
Now we will start to build the design from the bottom. Transistors will be used to build small cells, and these cells will then be used in bigger blocks, and so on. This means that the design will be built in a bottom-up fashion. After each step along the way, the design needs to be simulated to verify that each step work correctly.

\subsection{Phase 5: Chip assembly}
After the layout of the top cell is done, there is still some work to do. We need to add some circuits so that the block can communicate with the off-chip hardware, and after this, the final simulations need to be done.
